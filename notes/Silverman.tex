\documentclass{article}
\usepackage[utf8]{inputenc}

\usepackage{geometry}
\usepackage{amsmath}
\usepackage{tikz}
\usepackage{pgfplots}
\usepackage{graphicx}
\usepackage{graphics}
\usepackage{float}
\usepackage{listings}
\usepackage{amssymb}
\usepackage{amsfonts}
\usepackage[english]{babel}
\usepackage[utf8]{inputenc}
\usepackage{fancyhdr}
\usepackage{amsthm}
\pagestyle{fancy}
\fancyhf{}
\rhead{Silverman}
\lhead{Calvin Roth}
\rfoot{Page \thepage}
\pgfplotsset{compat = 1.16}
\title{Graph Theory Homework 4}
\author{Calvin Roth}
\date{March 2020}
\vspace{1mm}
\begin{document}
\newtheorem{theorem}{Theorem}
\newtheorem{lem}{Lemma}[section]
\newtheorem{claim}[lem]{Claim}

\newcommand{\prob}[1]{\section{} \noindent \textbf{Statement} #1 $ $\\ \textbf{Solution} $ $\\ }
\newcommand{\soln}{\noindent \textbf{Solution} $ $\\ }
\newcommand{\sol}{\noindent \textbf{Solution} $ $}

\newcommand{\IC}[1]{\noindent \textbf{Inductive Case} #1 $ $ \\}
\newcommand{\BC}[1]{\noindent \textbf{Base Case} #1 $ $ \\}

\newcommand{\inner}[1]{\langle #1 | #1\rangle}
\newcommand{\R}{\mathbb{R}}
\newcommand{\Z}{\mathbb{Z}}
\newcommand{\N}{\mathbb{N}}
\newcommand{\C}{\mathbb{C}}
\newcommand{\Q}{\mathbb{Q}}

\newcommand{\curve}[1]{C($\#1$)}

\newcommand{\set}[1]{\{$#1$\}}
\newcommand{\setc}[2]{\{$#1$ : $#2$\}}
\newcommand{\term}[2]{\textbf{Definition} \textit{#1} #2 $ $ \\}
\makeatletter
\renewcommand{\@seccntformat}[1]{Chapter \csname the#1\endcsname\quad}
\makeatother

It is time for so the bring out the heavy machinery of factoring, elliptic curves. It will take some time to development but it will be worth the investment. Firstly because elliptic curves are elegant and mathematically interesting in their own right but also are the premiere method for factor large numbers.

The study of elliptic curves involves roots of a specific class of cubic polynomials. Specifically, we care about cubics of the form: \begin{equation}
    ax^3 +  bx^2 y + c y^2 x + d y^3 + e x^2 + f x y + g y^2 + h x + i y + j = 0
\end{equation}
where all of the coefficients are rational numbers. 

\newpage
\section{Geometry and arithmetic}
\term{Rational line}{A line is called rational if it can be written as $ax + by + c = 0$ where a,b,c are rationals}


Next they do some cute proofs little describing all the triples a,b,c $\in \mathbb{Z}$ s.t. $a^2 + b^2 = c^2$
Idea:
Write $x' = \frac{x}{Z}$ and $y' = \frac{y}{Z}$
Then we can use a stereographic projection to derive the result.

It's a nice introduction.


Big theorem of book, Mordell's theorem: if C is a non-singular rational curve then the rational points of these curve is finitely and can be generated by taking intersections of lines. 



A curve having a solutions mod m does not imply the curve not modded has any solutions. 

Suppose we have a point on the curve, lets call it O. Given two more points P Q, we define the operation $P + Q := O * (P * Q)$ where * is the operation of drawing a line between the two points and taking the intersection. This is a group. Commutative even.


The + operation can be defined for any rational point O. 

\term{Weierstrass Form}{A cubic is in Weierstrass form if it is of the form $y^2 = 4x^3 + g_2 x + g_3$ or $y^2 = x^3 + a x^2 + b x + c $}

We can always do this. Silverman shows the procedure, idea is to change axes in a nice way. It isn't obvious but this is actually a homomorphism, which is interesting because we have defined our operations geometrically but have changed the geometry. 

\term{Elliptic Curve}{A cubic curve that has distinct complex roots}


There is a point at infinity, and it is an inflection point. We will call this point O. We also pretend it is rational. 

Let $P_1 = (x\_1 , y_1), P_2 = (x_2, y_2)$ where $x_1 \neq x_2$. Then we can define $P_1 + P_2 = (x_3, y_3)$ to be with $\lambda = \frac{y_2 - y_1}{x_2 - x_1}$ and $v = y_1 - \lambda x_1$ $x_3 = \lambda^2 - a - x_2 - x_1$ and $y_3 = \lambda x\_3 + v$ 

If we want to calculate the x coordinate of 2P we can use the $\textbf{Duplication Formula}$ given by $\frac{x^4 - 2 b x^2 - 8cx + b^2 - 4ac}{4x^3 + 4ax^2 + 4bx + 4c}$
 
 
 \section{Points of Finite Order}
 
\term{Order m}{In an elliptic a point P has order m is m is the least positive integer s.t. mP = O}

Plan look at how things be in the complex world and then say what that says about the real case
Since points of order 2 satisfy $(x, y) = P = -P = (x, -y)$ only those points of the form $(x,0)$ can have order 2. Specifically, ther are at maximum 3 points, corresponding to the three complex roots of the equation. 

The points of order three satisfy the condition $x(2P) = x(P)$ using the duplication formula we get that this condition amounts x is a root of :
$$ 3x^4 + 4ax^3 + 6bx^2 + 12cx +  4ac - b^2 $$

This amounts to nine possibly solutions: $O, (x, \pm y)$ each of the 4 roots of this equation. In $\R$ there is either a cyclic group of order 3 or the trivial group. 

\term{$C(\Q)$}{ $C(\Q)$ = \setc{(x,y)\in C}{x,y \in \Q} $\bigcup O$   } and likewise for $\Q, \C, \R$

We have that \set{O} $\subset C(\Q) \subset C(\R) \subset C(\C)$


We also have that \curve{R} is a compact 1D lie group. It is a known fact that compact 1d lies groups are isomorophic to rotations of a circle. This tells us that the points of finite order somehow correspond to roots of unity in the circle picture. Therefore if C is one component then the points of order m form a cyclic group of order m. But if C is two pieces there are two possibilities for the points of order m. If m is odd then we have a cyclic group of order m otherwise if m is even we have the direct product of a group of order 2 and a cyclic group of order m. 

Let's go back to \curve{Q}. Let's rewrite our curve in the form $y^2 = 4x^3 + g_2 x + g_3$, it is shown that when-
ever you have two complex numbers $g_2$ and $g_3$ so that the polynomial $4x^3 - g_2x - g_3$ has distinct roots, i.e., such that $g_2^3 - 27g_3^2 = 0$ $\omega_1 , \omega_2$ you can talk about lattices. I skimmed this. 


We can rewrite $y^2 = x^3 + a x^2 + b x + c $ as $Y^2 = X^3 + d^2 a X_2 + d^4 bX + d^6 c$ where $X = d^2 x$ and $Y = d^3 y$. We do this so that with large enough d, all the coefficients are integers. 

\begin{theorem}[Nagell–Lutz]
     If P=(x,y) is a point of finite order on the curve above, then x and y are integers. Furthermore either y=0 and we have a point of order two or $y | D$ where y is the discriminant. 
\end{theorem}

Idea to prove second part assuming the first is to write D as $D = r(x) f(x) + s(x) f'(x)$ where r and s are unspecified polynomials. 

The trickier statement is that these points are integers. For a prime p, We can always write the rational number $\frac{a}{b} = \frac{m}{n}p^v$ where $gcd(m,p) = gcd(n,p) = 1$. $Ord(\frac{a}{b}) = v$

\begin{theorem}[Nagell–Lutz Strong]
     If P=(x,y) is a point of finite order on the curve above, then x and y are integers. Furthermore either y=0 and we have a point of order two or $y^2 | D$ where y is the discriminant. 
\end{theorem}

\begin{theorem}[Mazur]
     Let C be a non-singular rational cubic curve, and suppose that \curve{Q} contains a point of finite order m. Then either $1 \leq m \leq 10$ om = 12.
More precisely, the set of points of finite order in \curve{Q} forms a subgroup that
has one of the following forms:
(i) A cyclic group of order N with $1 \leq N \leq 10$ or N = 12.
(ii) The product of a cyclic group of order two and a cyclic group of order 2N
with $1 \leq N \leq 4$.
\end{theorem}

\section{The group of rational points}
\term{Height}{The height of a rational number in lowest terms is $H(\frac{m}{n}) = max\set{|m|,|n|}$}
For a point P on a curve $H(P) = H(x)$. Furthermore $h(P) = \log_2 (P)$. Mordell's follows from 4 important lemmas which are proved later.

\begin{lem}
For every real number M \setc{P \in C(\Q) }{h(P) \leq M} is finite
\end{lem}

\begin{lem}
Let $P_0$ be a rational point of C. There exists a constant k(which depends on a,b,c, and k) such that $h(P_0 + P) \leq 2 h(P) + k$ forall points P.
\end{lem}

\begin{lem}
There is a constant l such that $h(2P) \leq 4h(P) - l$
\end{lem}

\begin{lem}
The index $(C(\Q) : 2C(\Q) )$ is finite.
\end{lem}

As a general matter of algebra having a function from $G \to [0,\infty)$ with these properties implies the group is finitely generated.
The proof when we put these ideas together is a proof by desencent.

We know that a that a group is finitely generated we may write $G \cong \Z \oplus \Z \cdots \oplus \Z \oplus \Z_{p_1^{v_1}} \oplus \cdots \oplus \Z_{p_n^{v_n}}$

The number of $\Z$ is called the rank. The subgroup with only finite parts is called the torision subgroup. 

\section{Cubic Curves over finite fields}

Notes Coming Soon
\end{document}
